\documentclass[10pt,a4paper]{article}
\usepackage[utf8]{inputenc}
\usepackage[german]{babel}
\usepackage{amsmath}
\usepackage{amsfonts}
\usepackage{amssymb}

\newcommand{\entspricht}{\mathop{\widehat{=}}}

\begin{document}
\begin{center}
\textbf{Stochastik 1 \\ Serie 2 \\}
\end{center}

\begin{flushright}
Kevin Stehn 6416016 Gruppe 3 \\
Konstantin Kobs 6414943 Gruppe 2
\end{flushright}

\section*{Aufgabe 1}
$\Omega = \{(\omega_{1},\omega_{2},\omega_{3}) | \omega_i \in \{ \text{1, 2, \dots, 12}\} \wedge \omega_1 < \omega_2 < \omega_3 \}$\\ mit $\text{rot} \entspricht \{1, 2, 3, 4, 5\}, \text{gelb} \entspricht \{6, 7, 8\}, \text{grün} \entspricht \{9, 10, 11, 12\}$.\\
Das Ereignis, dass alle drei die gleiche Farben haben, ist:\\
$A = \{(\omega_1, \omega_2, \omega_3) \in \Omega | \text{für alle } i \in \{1,2,3\} \text{ gilt: } \omega_i \text{ entweder} \in \text{rot, gelb oder grün} \}$ unter der Voraussetzung, dass die oben beschriebenen Zahlenmengen mit der Farbe bezeichnet werden.\\
Wir berechnen die Mächtigkeiten der Mengen mit Hilfe des Binomialkoeffizienten, da wir nicht ``zurücklegen'' und die Reihenfolge nicht beachten:
\begin{align*}
	|\Omega| &= \binom{12}{3} = 220\\
	|A| &= \binom{5}{3} + \binom{3}{3} +\binom{4}{3} = 15
\end{align*}
Damit ist die Wahrscheinlichkeit, drei mal die selbe Farbe zuf\"allig zu ziehen: \\
$$P(A) = \frac{|A|}{|\Omega|} = \frac{15}{220} = \frac{3}{44} = 0,06\overline{81}$$

\section*{Aufgabe 2}
$\Omega = \{(\omega_{1},\omega_{2},\omega_{3},\omega_{4},\omega_{5},\omega_{6},\omega_{7}) | \omega_{i} \in \{1,...,7\} \forall i \neq j : \omega_{i} \neq \omega_{j}\}$ \\
$|\Omega| = 7!$, da wir jede Stelle ``ohne Zurücklegen'' befüllen können.

\begin{itemize}
\item[(a)] $A = \{(\omega_{1},\omega_{2},\omega_{3},\omega_{4},\omega_{5},\omega_{6},\omega_{7}) \in \Omega | \omega_{7} \in \{2,4,6\}  \}$\\
Die Mächtigkeit dieser Menge beträgt $3 \cdot 6!$, da wir eine Stelle mit einer der drei Zahlen reservieren und die restlichen Stellen beliebig füllen können (aber ``ohne Zurücklegen''). Die Wahrscheinlichkeit beträgt also $\frac{3 \cdot 6!}{7!} = \frac{3}{7} \approx 0,429$
\item[(b)] $B = \{(\omega_{1},\omega_{2},\omega_{3},\omega_{4},\omega_{5},\omega_{6},\omega_{7}) \in \Omega |\sum \omega_{i} \forall i \in\{1,...,7\} \text{ ist durch 3 teilbar} \}$\\
Dieses Ereignis hat die Wahrscheinlichkeit null, da jede Zahl aus $\Omega$ die gleiche Quersumme besitzt. Diese ist 28 und somit nicht durch 3 teilbar. Damit ist auch jede Zahl aus $\Omega$ nicht durch drei teilbar.
\end{itemize}

\section*{Aufgabe 3}
Wir haben $\binom{36}{12}$ verschiedene Klausuren zu betrachten, da wir aus dem Aufgabenpool 12 Aufgaben ohne Zurücklegen und ohne Beachtung der Reihenfolge ziehen. Dabei seien die Aufgaben von 1 bis 36 durchnummeriert.\\
$\Omega = \{(\omega_{1},...,\omega_{12}) | \omega_{i} \in \{1,...,36\} \wedge \omega_{i} < \omega_{j} \}$ \\
Angenommen, der Student kann die ersten 20 Aufgaben lösen. Dann gilt für das Ereignis A ``Der Student besteht die Prüfung'':\\
$A = \{(\omega_{1},...,\omega_{12}) \in \Omega | \text{Anzahl der }\omega_{i} \in \{1,...,20\} \geq 6 \}$ \\
Nun müssen wir die Einzelwahrscheinlichkeiten summieren, dass der Student \textit{genau} 6, 7, 8, 9, 10, 11 oder 12 Aufgaben richtig beantwortet. Das ist ähnlich der Überlegung, beim Lotto ``genau drei Richtige zu bekommen''. Es gilt also:\\
\begin{align*}
	P(A) &= \sum_{i=6}^{12} \frac{\binom{20}{i} \cdot \binom{16}{12 - i}}{\binom{36}{12}}\\
	&= \frac{247}{310}\\
	&\approx 0,797
\end{align*}

\end{document}