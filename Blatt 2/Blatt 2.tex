\documentclass[10pt,a4paper]{article}
\usepackage[utf8]{inputenc}
\usepackage[german]{babel}
\usepackage{amsmath}
\usepackage{amsfonts}
\usepackage{amssymb}

\newcommand{\entspricht}{\mathop{\widehat{=}}}

\begin{document}
\begin{center}
\textbf{Stochastik 1 \\ Serie 2 \\}
\end{center}

\begin{flushright}
Kevin Stehn 6416016 Gruppe 3 \\
Konstantin Kobs 6414943 Gruppe 2
\end{flushright}

\section*{Aufgabe 1}
$\Omega = \{(\omega_{1},\omega_{2},\omega_{3}) | \omega_{i} \in \{\{1,...,12\} \vee \omega_{1}<\omega_{2}<\omega_{3} \}$ \\
F\"ur $\omega_{i} = 1,2,3,4,5 \entspricht$ rot \\
 	  $\omega_{i} = 6,7,8 \entspricht$ gelb\\
  	  $\omega_{i} = 9,10,11,12 \entspricht$ gr\"un \\
Das Ereignis das alle drei die gleiche Farben haben ist:\\
$A = \{(\omega_{1},\omega_{2},\omega_{3}) \in \Omega |\omega_{1}=\omega_{2}=\omega_{3}=rot\vee gelb \vee $gr\"un$\}$
Daraus lassen sich drei Teilereignisse Bilden\\
$A_{rot} =\{(\omega_{1},\omega_{2},\omega_{3}) \in \Omega |\omega_{1}=\omega_{2}=\omega_{3}=rot\}$ \\
$A_{gelb} =\{(\omega_{1},\omega_{2},\omega_{3}) \in \Omega |\omega_{1}=\omega_{2}=\omega_{3}=gelb\}$\\
$A_{gruen} =\{(\omega_{1},\omega_{2},\omega_{3}) \in \Omega |\omega_{1}=\omega_{2}=\omega_{3}=gruen\}$ \\
Da wir ohne Beachtung auf die Reihenfolge und ohne zur\"ucklegen ist die M\"achtigkeit der Mengen wie folgt:\\
$|\Omega| = \binom{12}{3} = 220$ \\
$|A| = |A_{rot}| + |A_{gelb}| + |A_{gruen}| = \binom{5}{3} + \binom{3}{3} +\binom{4}{3} = 15$ \\
Damit ist die Wahrscheinlichkeit drei mal die selbe Farbe zuf\"allig zu ziehen: \\
$P(A) = \frac{|A|}{|\Omega|} = \frac{15}{220} = \frac{3}{44} = 0,06\overline{81}$

\section*{Aufgabe 2}
$\Omega = \{(\omega_{1},\omega_{2},\omega_{3},\omega_{4},\omega_{5},\omega_{6},\omega_{7}) | \omega_{i} \in \{1,...,7\} \forall i \in \{1,...,7\}: \omega_{i} \neq \omega_{j}\}$ \\

\begin{itemize}
\item[(a)] $A = \{(\omega_{1},\omega_{2},\omega_{3},\omega_{4},\omega_{5},\omega_{6},\omega_{7}) \in \Omega | \omega_{7} \in \{2,4,6\}  \}$
\item[(b)] $B = \{(\omega_{1},\omega_{2},\omega_{3},\omega_{4},\omega_{5},\omega_{6},\omega_{7}) \in \Omega |\sum \omega_{i} \forall i \in\{1,...,7\} ist durch 3 Teilbar \}$ Dieses Ereignis hat die Wahrscheinlichkeit null, da es nur eine Quersumme gibt und diese ist nicht durch 3 Teilbar da diese 28 ist.
\end{itemize}

\section*{Aufgabe 3}
$\Omega = \{(\omega_{1},...,\omega_{12}) | \omega_{i} \in \{1,...,36\} \vee \omega_{i} \neq \omega_{j} \}$ \\
$A = \{(\omega_{1},...,\omega_{6}) \in \Omega|\omega_{i} \in \{1,...,20\} \vee \omega_{i} \neq \omega_{j} \}$ \\

$|\Omega| = \binom{36}{12} $ \\
$|A| = \binom{20}{6}$ \\

$P(A) = \frac{|A|}{|\Omega|} = \frac{\binom{20}{6}}{\binom{36}{12}}= \frac{38}{1227135} = 3,097*10^{-5}$
\end{document}