\documentclass[10pt,a4paper]{article}
\usepackage[utf8]{inputenc}
\usepackage[german]{babel}
\usepackage{amsmath}
\usepackage{amsfonts}
\usepackage{amssymb}

\newcommand{\ent}{\mathop{\widehat{=}}}

\begin{document}
\begin{center}
\textbf{Stochastik 1 \\ Serie 11 \\}
\end{center}

\begin{flushright}
Kevin Stehn 6416016 Gruppe 3 \\
Konstantin Kobs 6414943 Gruppe 2
\end{flushright}

\section*{Aufgabe 1 und 2}
\begin{itemize}
\item[(a)]
Damit $f$ eine Wahrscheinlichkeitsdichte ist, muss gelten:
$$\int_{-\infty}^{\infty}f(t) dt = 1$$
Daraus folgt für c:
\begin{align*}
\int_{-\infty}^{\infty}f(t) dt &= \int_{-\infty}^{0}f(t) dt + \int_{0}^{\infty}f(t) dt\\
&= \int_{0}^{\infty} cte^{-t/2} dt\\
&= c \cdot \int_{0}^{\infty} te^{-t/2} dt\\
&= c \cdot \left( [-2te^{-t/2}]_0^{\infty} - \int_0^\infty -2e^{-t/2} dt \right)\\
&= c \cdot \left( [-2te^{-t/2}]_0^{\infty} - [ 4e^{-t/2} ]_0^\infty \right)\\
&\overset{\text{l'hopital}}{=} c \cdot \left((0 - 0) - ( 0 - 4 ) \right)\\
&= 4c = 1 \Rightarrow c = \frac{1}{4}
\end{align*}

\item[(b)]
Zu berechnen ist:
$$\int_{-\infty}^{x}f(t) dt$$
Dabei muss unterschieden werden zwischen:
\item[-] $x \leq 0$: Dann ergibt das oben gegebene Integral immer $0$.
\item[-] $x > 0$:
\begin{align*}
\int_{0}^{x}f(t) dt &= \int_{0}^{x} \frac{1}{4}te^{-t/2} dt\\
&= \frac{1}{4} \cdot \int_{0}^{x} te^{-t/2} dt\\
&= \frac{1}{4} \cdot \left( [-2te^{-t/2}]_0^x - \int_{0}^{x} -2e^{-t/2} dt\right)\\
&= \frac{1}{4} \cdot \left( [-2te^{-t/2}]_0^x - [4e^{-t/2}]_0^x \right)\\
&= \frac{1}{4} \cdot \left( -2xe^{-x/2} - (4e^{-x/2} - 4) \right)\\
&= \frac{1}{4} \cdot \left( -2xe^{-x/2} - 4e^{-x/2} + 4 \right)\\
&= -\frac{1}{2}xe^{-x/2} - e^{-x/2} + 1\\
&= \left( -\frac{1}{2}x - 1\right) \cdot e^{-x/2} + 1\\
\end{align*}

Somit ist die Verteilungsfunktion die folgende:
\begin{equation*}
F(x) =
\left\{
	\begin{array}{ll}
		0  & \mbox{wenn } x \leq 0 \\
		\left( -\frac{1}{2}x - 1\right) \cdot e^{-x/2} + 1 & \mbox{wenn } x > 0
	\end{array}
\right.
\end{equation*}

\item[(c)]
\begin{align*}
P(X \geq 6) &= 1 - P(X \leq 6) + P(\{6\})\\
&= 1 - F(6) + P(\{6\})\\
&= 1 - (-4 \cdot e^{-3} +1) + 0\\
&= 4 e^{-3} \approx 0,1991
\end{align*}
\end{itemize}

\section*{Aufgabe 3}
\begin{itemize}
\item[(a)]
Um von der Dichtefunktion von $X$ auf die Dichtefunktion von $-X$ zu kommen, müssen die Werte $f(x)$ und $f(-x)$ für alle $x \in \mathbb{R}$ vertauscht werden. Da diese aber laut Aufgabe identisch sind, ist die Dichtefunktion von $X$ identisch mit der Dichtefunktion von $-X$. Dadurch ist auch die Verteilungsfunktion identisch. $\square$

\item[(b)]
Der Erwartungswert berechnet sich wie folgt:
$$\int_{-\infty}^{\infty} tf(t) dt$$

Wir fassen zusammen und addieren die linke und rechte Seite an den Stellen $t$ und $-t$ für alle $t \in \mathbb{R}^+$:
$$t \cdot f(t) + (-t) \cdot f(-t)$$

Da $f(-t) = f(t)$ für alle $t \in \mathbb{R}$ gilt, können wir schreiben:
$$t \cdot f(t) + (-t) \cdot f(t) = f(t) \cdot (t-t) = 0$$
Es heben sich also alle Werte links und rechts auf.\\
Für $t=0$ gilt dann:
$$0 \cdot f(0) = 0$$
Alles zusammen ergibt dann also $0$, was das Ergebnis und damit unser Erwartungswert ist.

\end{itemize}
\end{document}