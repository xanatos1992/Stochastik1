\documentclass[10pt,a4paper]{article}
\usepackage[utf8]{inputenc}
\usepackage[german]{babel}
\usepackage{amsmath}
\usepackage{amsfonts}
\usepackage{amssymb}

\newcommand{\ent}{\mathop{\widehat{=}}}

\begin{document}
\begin{center}
\textbf{Stochastik 1 \\ Serie 8 \\}
\end{center}

\begin{flushright}
Kevin Stehn 6416016 Gruppe 3 \\
Konstantin Kobs 6414943 Gruppe 2
\end{flushright}

\section*{Aufgabe 1}
\begin{itemize}
\item[(a)] Unser Modell ist $\Omega = \{1,2,3,4,5\}^3$ und $P \ent Laplacemaß$. Die Zufallsgröße Z definieren wir mit $Z: \Omega \rightarrow \{1,2,3,4,5\}$ als kleinste gezogene Zahl.\\
Weil die Zahl $k$ \textit{mindestens} die kleinste Zahl der drei gezogenen Zahlen sein soll, können wir die Wahrscheinlichkeiten für $Z \geq k$ wie folgt aufstellen:
\begin{align*}
P(Z \geq 1) &= \frac{5^3}{5^3} = 1\\
P(Z \geq 2) &= \frac{4^3}{5^3} = \frac{64}{125}\\
P(Z \geq 3) &= \frac{3^3}{5^3} = \frac{27}{125}\\
P(Z \geq 4) &= \frac{2^3}{5^3} = \frac{8}{125}\\
P(Z \geq 5) &= \frac{1^3}{5^3} = \frac{1}{125}\\
\end{align*}

\item[(b)] 
$$E(Z) = \sum_{i=1}^5 P(Z \geq i) = 1 + \frac{64}{125} + \frac{27}{125}+ \frac{8}{125} + \frac{1}{125} = 1,8$$
\end{itemize}

\section*{Aufgabe 2}


\section*{Aufgabe 3}

\end{document}