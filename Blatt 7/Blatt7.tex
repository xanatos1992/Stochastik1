\documentclass[10pt,a4paper]{article}
\usepackage[utf8]{inputenc}
\usepackage[german]{babel}
\usepackage{amsmath}
\usepackage{amsfonts}
\usepackage{amssymb}

\newcommand{\ent}{\mathop{\widehat{=}}}

\begin{document}
\begin{center}
\textbf{Stochastik 1 \\ Serie 7 \\}
\end{center}

\begin{flushright}
Kevin Stehn 6416016 Gruppe 3 \\
Konstantin Kobs 6414943 Gruppe 2
\end{flushright}

\section*{Aufgabe 1 und 2}
\begin{itemize}
\item[(a)] Für k = 3 gilt $P(X=3) = \frac{3^3}{3!}\cdot e^{-3} \approx 0,224$ \\
Für k = 4 gilt schon $P(X=4) = \frac{3^4}{4!}\cdot e^{-4} \approx 0,168$. Daran sieht man das die Wahrscheinlichkeit aber k = 3 nur noch kleiner wird und es am Wahrscheinlichsten ist das 3 Schiffe die Schleuse anlaufen werden.
\item[(b)] Wenn die Schleuse 4 Schiffe abfertigen kann können wir die Wahrscheinlichkeit das min. ein Schiff nicht geschleust wird daraus berechnen, in dem wir die Gesamtwahrscheinlichkeit von 1 minus den Elementaren Wahrscheinlichkeiten von 1 bis 4 berechnen. Daraus ergibt sich\\
$P(X>=5) = 1- ( P(X=1)+ P(X=2)+P(X=3)+P(X=4))\newline
= 1- (\frac{3^1}{1!}\cdot e^{-3} + \frac{3^2}{2!}\cdot e^{-3} +\frac{3^3}{3!}\cdot e^{-3} + \frac{3^4}{4!}\cdot e^{-3} \newline
= 1-(e^{-3} \cdot (\frac{3^1}{1!} +\frac{3^2}{2!} +\frac{3^3}{3!} +\frac{3^4}{4!}))\approx 0,234$
\item[(c)] Der Erwartungswert berechnet sich wie folgt:\\
$E(X)= 1\cdot P(\{1\})+2\cdot P(\{2\})+3\cdot P(\{3\})+4\cdot P(\{4\}) \newline
= 1\cdot 0,149 + 2\cdot 0,224+3\cdot 0,224+4\cdot 0,168 \approx 1,941$
\end{itemize}


\section*{Aufgabe 3}
\begin{itemize}
\item[(a)] Damit X und Y unkorreliert sind muss gelten: $Cov(X,Y) = 0$ \\
Da Laplacemaß ist hat jedes $\omega \in \Omega$ die gleiche Wahrscheinlichkeit, also $\frac{1}{3}$.\\
$Cov(X,Y) = E\langle((X-E(X))*(Y-E(Y))\rangle \\
= \sum_{x,y} (x-E(X))*(y-E(Y)) * P(X=x,Y=y) \text{ das x und y können wir auseinander ziehen }\\
(a*)= \sum_x(x-E(X))*P(X=x)*\sum_y(y-E(Y))*P(Y=y)$ \text{ eine Summe null dann fertig}\\
Berechnen wir also erst mal den Erwartungswert von X:\\
$E(X) = 1*P(X=1)+0*P(X=2)-1*P(X=3) = \frac{1}{3} - \frac{1}{3} = 0$\\
Damit also weiter machen von (a*):\\
$=(1-0*\frac{1}{3}+0-0*\frac{1}{3}-1-0*\frac{1}{3})*\sum_y(y-E(Y))*P(Y=y) \\
= 0 * \sum_y(y-E(Y))*P(Y=y) = 0$.\\
Damit sind X und Y unkorreliert.
\item[(b)]
\end{itemize}
\end{document}