\documentclass[10pt,a4paper]{article}
\usepackage[utf8]{inputenc}
\usepackage[german]{babel}
\usepackage{amsmath}
\usepackage{amsfonts}
\usepackage{amssymb}

\newcommand{\ent}{\mathop{\widehat{=}}}

\begin{document}
\begin{center}
\textbf{Stochastik 1 \\ Serie 7 \\}
\end{center}

\begin{flushright}
Kevin Stehn 6416016 Gruppe 3 \\
Konstantin Kobs 6414943 Gruppe 2
\end{flushright}

\section*{Aufgabe 1 und 2}
\begin{itemize}
\item[(a)] Wir stellen eine Wertetabelle auf, um die Wahrscheinlichkeiten vergleichen zu können. Dabei erkennen wir, dass sowohl für $X = 2$ als auch für $X = 3$ die gleiche Wahrscheinlichkeit besteht. Diese beträgt $P(X = 2) = P(X = 3) = \frac{3^3}{3!}\cdot e^{-3} \approx 0,224$. Die Wahrscheinlichkeiten nehmen rundherum ab. Da wir wissen, dass die Wahrscheinlichkeiten poissonverteilt sind, sind diese die beiden Werte mit der höchsten erreichbaren Wahrscheinlichkeit.

\item[(b)] Die Frage ist nach $P(X > 4)$. Diese Wahrscheinlichkeit können wir über das Gegenereignis leichter bestimmen, nämlich mit $P(X > 4) = 1 - P(X \leq 4)$. Daraus ergibt sich:
\begin{align*}
P(X > 4) &= 1 - P(X \leq 4)\\
&= 1 - (P(X = 0) + P(X = 1) + P(X = 2) + P(X = 3) + P(X = 4))\\
&= 1- \left(\frac{3^0}{0!}\cdot e^{-3} + \frac{3^1}{1!}\cdot e^{-3} + \frac{3^2}{2!}\cdot e^{-3} +\frac{3^3}{3!}\cdot e^{-3} + \frac{3^4}{4!}\cdot e^{-3}\right)\\
&= 1 - e^{-3} \cdot (1 + 3 + 4,5 + 4,5 + 3,375)\\
&= 1 - \frac{16,375}{e^3}\\
&\approx 0,185
\end{align*}
Also muss mit etwa 18,5 prozentiger Wahrscheinlichkeit mindestens ein Schiff warten.

\item[(c)] Der Erwartungswert berechnet sich wie folgt:
\begin{align*}
E(X) &= 1 \cdot P(X = 1)+ 2 \cdot P(X = 2) + 3 \cdot P(X = 3) + 4 \cdot P(X = 4)\\
&= 1 \cdot 0,149 + 2 \cdot 0,224 + 3 \cdot 0,224 + 4 \cdot 0,168\\
&\approx 1,942
\end{align*}
\end{itemize}

\section*{Aufgabe 3}
\begin{itemize}
\item[(a)] Damit X und Y unkorreliert sind muss gelten: $Cov(X,Y) = 0$ \\
\begin{align*}
Cov(X,Y) &= E[(X - E(X)) \cdot (Y - E(Y))]\\
E(X) &= 1 \cdot \frac{1}{3} + 0 \cdot \frac{1}{3} + (-1) \cdot \frac{1}{3} = 0\\
E(Y) &= 0 \cdot \frac{2}{3} + 1 \cdot \frac{1}{3} = \frac{1}{3}\\
Cov(X,Y) &= (1-0)(0-1/3)\frac{1}{3} + (0-0)(1-1/3)\frac{1}{3} + (-1-0)(0-1/3)\frac{1}{3}\\
&= 0
\end{align*}

\item[(b)] Damit die beiden Zufallsgrößen stochastisch unabhängig sind, müssten folgende Eigenschaften gelten
\begin{align*}
P(X=-1,Y=0) &= P(X=-1) \cdot P(Y=0)\\
P(X=-1,Y=1) &= P(X=-1) \cdot P(Y=1)\\
P(X=0,Y=0) &= P(X=0) \cdot P(Y=0)\\
P(X=0,Y=1) &= P(X=0) \cdot P(Y=1)\\
P(X=1,Y=0) &= P(X=1) \cdot P(Y=0)\\
P(X=1,Y=1) &= P(X=1) \cdot P(Y=1)
\end{align*}
da eine stochastische Unabhängigkeit zustande kommt, wenn die Wahrscheinlichkeiten aller Kombinationen der Zufallsgrößen Produktgestalt haben.

Dies lässt sich sehr leicht widerlegen, da schon die erste Gleichung nicht aufgeht. Es gilt:
$$P(X=-1,Y=0) = \frac{1}{3} \neq \frac{1}{3} \cdot \frac{2}{3} = P(X=-1) \cdot P(Y=0)$$
Damit ist gezeigt, dass die beiden Zufallsgrößen nicht stochastisch unabhängig sind.

\end{itemize}
\end{document}