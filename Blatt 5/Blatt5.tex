\documentclass[10pt,a4paper]{article}
\usepackage[utf8]{inputenc}
\usepackage[german]{babel}
\usepackage{amsmath}
\usepackage{amsfonts}
\usepackage{amssymb}

\newcommand{\ent}{\mathop{\widehat{=}}}

\begin{document}
\begin{center}
\textbf{Stochastik 1 \\ Serie 5 \\}
\end{center}

\begin{flushright}
Kevin Stehn 6416016 Gruppe 3 \\
Konstantin Kobs 6414943 Gruppe 2
\end{flushright}

\section*{Aufgabe 1}
\begin{itemize}
\item[(a)]Nachweis der Dichteeigenschaft:\\
1.
$$1 = \sum_{n=1}^\infty p_n = \sum_{n=1}^\infty \frac{c}{n^2} = c \cdot \sum_{n=1}^\infty \frac{1}{n^2} = c \cdot \frac{\pi^2}{6}$$
$$\Leftrightarrow c = \frac{6}{\pi^2}$$
2.\\
Da $c$ positiv ist, sind auch alle Werte der Folge $p_n$ für $n > 0$ positiv. Somit gilt auch diese Eigenschaft.
\item[(b)]
\begin{align*}
P(X \geq 3) &= 1 - P(X = 2) - P(X = 1)\\
&= 1 - \frac{6}{4\pi^2} - \frac{6}{\pi^2}\\
&\approx 0,24
\end{align*}
\end{itemize}

\section*{Aufgabe 2}
Das Model für unser Sonderausschuss ist:\\
$\Omega = \{ (\omega_1,...,\omega_5) | \omega_i \in \{A\}^5 \cup \{B\}^7 \text{ f\"ur } i= 1,...,5 \}$ und P = Laplacemaß da jedes Mitglied ein gleich hohe Wahrscheinlichkeit hat gezogen zu werden. Dabei gilt $\omega_i = A \ent \text{Vertreter i ist aus Gruppe A und } \omega_i = B \ent \text{Der Vertreter i ist aus Gruppe B}$.
Das Ereignis, dass 2 Vertreter aus Gruppe A im Ausschuss ist:\\
$A = \{\omega \in \Omega | \sum(\omega_i = A) = 2 \text{ f\"ur } i = 1,...,5\} =\newline \{(A,A,B,B,B),(A,B,A,B,B),(A,B,B,A,B),(A,B,B,B,A),\newline(B,A,B,B,A),(B,B,A,B,A),(B,B,B,A,A) \} \newline
|\Omega| = 95040 \text{ und  } |A| = 7$. \\
Daraus folgt die Wahrscheinlichkeit für das Ereignis A:\\
$P(A) = \frac{|A|}{|B|} = \frac{7}{95040} = 7.36 \cdot 10^{-5}$
\section*{Aufgabe 3}
\begin{itemize}
\item[(a)]Unser Modell: $\Omega = \{0,1\}^{100}$ dabei gilt $0 \ent \text{ nicht Geburtstag und } 1 \ent \text{ Geburtstag}$ Betrachten wir nun für alle i = 1,...,100:\\
$X_i: \Omega \rightarrow \{0,1\}$ dabei folgt $X_i \sim Ber(\frac{1}{365})$.
Daraus lässt sich die Zufallsgröße X wie folgt bilden: 
\[X = \sum_{i=1}^{100} X_i \]
\item[(b)]Da wir eine \grqq großes\grqq \text{ }n haben und ein \grqq kleines\grqq \text { } p, lässt es sich Poissonverteilt ansehen. Dabei ist der Parameter 
$\lambda = 100*\frac{1}{365} = \frac{20}{73} $
\item[(c)] $P(X=0) \approx \frac{\lambda^k}{k!}*e^{-\lambda} = \frac{\frac{20}{73}^0}{0!} *e^{-\frac{20}{73}} \approx 0,76$
\end{itemize}
\end{document}