\documentclass[10pt,a4paper]{article}
\usepackage[utf8]{inputenc}
\usepackage[german]{babel}
\usepackage{amsmath}
\usepackage{amsfonts}
\usepackage{amssymb}

\newcommand{\ent}{\mathop{\widehat{=}}}

\begin{document}
\begin{center}
\textbf{Stochastik 1 \\ Serie 10 \\}
\end{center}

\begin{flushright}
Kevin Stehn 6416016 Gruppe 3 \\
Konstantin Kobs 6414943 Gruppe 2
\end{flushright}

\section*{Aufgabe 1}
\begin{itemize}
\item[(a)]
$ F(x)=\left\{\begin{array}{cl} 0, & x < 0\\ 
\frac{1}{2}, & x = 0 \\ 
\frac{x}{60}, & x > 0 \\
1 , & x \geq 60 \end{array}\right.$
\item[(b)]$P(\{15\}) = P((-\infty,15] \setminus (-\infty,15)) \\
= F(15) - \lim\limits_{\epsilon \searrow 0 }{F(15-\epsilon)}\\
= \frac{15}{60} - \frac{15}{60} = 0$
\end{itemize}

\section*{Aufgabe 2}
Damit $F_c(x)$ eine Vf ist muss gelten: - Monoton steigen dies gilt \\
- rechtseitig stätig durch dich Bedingung gilt dies auch\\
- $\lim\limits_{x \rightarrow -\infty }{F(x)} = 0$ und $\lim\limits_{x \rightarrow \infty }{F(x)} = 1$ \\
Die Eigentschaft $\lim\limits_{x \rightarrow -\infty }{F(x)} = 0$ wird schon abgedeckt durch die Bedingung $x < 1$. Damit also $\lim\limits_{x \rightarrow \infty }{F(x)} = 1$ muss noch c gefunden werden.\\
$\lim\limits_{x \rightarrow \infty }{F(x)} = 1 \Leftrightarrow \lim\limits_{x \rightarrow \infty }{c \cdot(1-\frac{1}{x})} = 1 \Leftrightarrow c \cdot \lim\limits_{x \rightarrow \infty }{ (1-\frac{1}{x})} = 1$ \\
$\frac{1}{x} $ geht gegen null damit ist\\
$c \cdot (1-0) = 1 \Leftrightarrow c = 1$.
Das heißt damit F eine Vf ist muss c = 1 sein.
\section*{Aufgabe 3}
$ F(x)=\left\{\begin{array}{cl} 0, & x \leq 0\\ 
\frac{\lambda^x}{x!}*e^{-\lambda}, & x > 0  \end{array}\right.$\\
Damit sind auch alle Eigenschaften abgedeckt.
\end{document}