\documentclass[10pt,a4paper]{article}
\usepackage[utf8]{inputenc}
\usepackage[german]{babel}
\usepackage{amsmath}
\usepackage{amsfonts}
\usepackage{amssymb}

\newcommand{\ent}{\mathop{\widehat{=}}}

\begin{document}
\begin{center}
\textbf{Stochastik 1 \\ Serie 10 \\}
\end{center}

\begin{flushright}
Kevin Stehn 6416016 Gruppe 3 \\
Konstantin Kobs 6414943 Gruppe 2
\end{flushright}

\section*{Aufgabe 1}
\begin{itemize}
\item[(a)]
$ F(x)=\left\{\begin{array}{cl} 0, & x < 0\\ 
\frac{1}{2}, & x = 0 \\ 
\frac{1}{2}+\frac{x}{120}, & x > 0 \\
1 , & x \geq 60 \end{array}\right.$
\item[(b)]
\begin{align*}
P([15, \infty)) &= P((-\infty,\infty) \setminus (-\infty,15)) \\
&= 1 - F(15)\\
&= 1 - 0,625\\
&= 0,375
\end{align*}
\end{itemize}

\section*{Aufgabe 2}
Damit $F_c(x)$ eine Verteilungsfunktion ist, muss gelten:
\begin{itemize}
\item[-]$\lim\limits_{x \rightarrow \infty }{F(x)} = 1$ \qquad Es muss also $\lim\limits_{x \rightarrow \infty }{c \cdot (1-\frac{1}{x})} = 1$ sein. Für $x \rightarrow \infty$ gilt $\frac{1}{x}=0$, sodass $c \cdot (1-0) = 1$ sein muss. Damit ist $c=1$.
\item[-]$\lim\limits_{x \rightarrow -\infty }{F(x)} = 0$ \qquad Da alle Wahrscheinlichkeiten für $x < 1$ $0$ sind, stimmt diese Aussage.
\item[-]monoton steigend \qquad Beide Teilabschnitte der Funktion sind monoton steigend. Der kleinste mögliche Wert des zweiten Abschnittes ist genau $0$, sodass der Graph der Funktion an jeder Stelle höchstens größer wird.
\item[-]rechtsseitige Stetigkeit \qquad Die Funktion ist in allen Punkten stetig, weshalb diese Eigenschaft implizit erfüllt ist.
\end{itemize}
\section*{Aufgabe 3}
$$F(x)=\left\{\begin{array}{cl} 0, & x < 0\\ 
\frac{\lambda^x}{x!} \cdot e^{-\lambda}, & x \geq 0  \end{array}\right.$$
Damit sind auch alle Eigenschaften abgedeckt.\\
\end{document}