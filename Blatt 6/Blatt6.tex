\documentclass[10pt,a4paper]{article}
\usepackage[utf8]{inputenc}
\usepackage[german]{babel}
\usepackage{amsmath}
\usepackage{amsfonts}
\usepackage{amssymb}
\usepackage{eurosym}

\newcommand{\ent}{\mathop{\widehat{=}}}

\begin{document}
\begin{center}
\textbf{Stochastik 1 \\ Serie 6\\}
\end{center}

\begin{flushright}
Kevin Stehn 6416016 Gruppe 3 \\
Konstantin Kobs 6414943 Gruppe 2
\end{flushright}

\section*{Aufgabe 1}
Wir Behaupten: $E(X+Y) = E(X) + E(Y)$\\
Beweis dass dies stimmt:\\
$E(X+Y) = \sum_{x\in E} \sum_{y \in E} x \cdot P(X=x) + y \cdot P(Y=y)\\
\\ = \sum_{x\in E} x \cdot P(Y=x) + \sum_{y \in E} y\cdot P(Y=y) 
\\\\= E(X) + E(Y)$
\section*{Aufgabe 2}
\begin{itemize}
\item[(a)]
\item[(b)]
\item[(c)]
\end{itemize}

\section*{Aufgabe 3}
Unser Modell sieht wie folgt aus $\Omega = \{\{1,...,6\}^2\} \text{ } P = Bin(1,\frac{6}{36})$. Wir k\"onnen Bernoulli hier verwenden da wir nur die M\"oglichkeit Pasch und kein Pasch betrachten, dabei ist Pasch der Erfolg. Unsere Zufallsgröße ist:$X: \Omega \rightarrow \{-3,0,2\}$. Dabei gilt: $-3 \ent \text{3 \euro Verlust} \text{ } 0 \ent \text{kein Gewinn } 2 \ent \text{2 \euro Gewinn}$
\begin{align*}
X(\omega) = \left\{
\begin{array}{l l}
& -3, \text{ f\"ur } \omega = (6,6)  \\ 
& 0 , \text{ f\"ur }  \omega \in \{(x,y)|x=y \text{ und } x,y \in \{1,...,5\}\\
&  2, \text{ f\"ur } \omega \in \{(x,y)|x \neq y \text{ und } x,y \in \{1,...,6\}
\end{array}
\right.
\end{align*} \\
$E(X) = \sum_{x\in E} x\cdot P(X=x) \\
= -3*(P=-3) + 0*P(X=0)+2*P(X=2)\\\\
=-3*\binom{1}{1}*\frac{1}{36}^1+2*\binom{1}{1}*\frac{30}{36}^1\\\\
= \frac{19}{12} \approx 1,58\text{\euro}$
\end{document}