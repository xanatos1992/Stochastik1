\documentclass[10pt,a4paper]{article}
\usepackage[utf8]{inputenc}
\usepackage[german]{babel}
\usepackage{amsmath}
\usepackage{amsfonts}
\usepackage{amssymb}
\usepackage{eurosym}

\newcommand{\ent}{\mathop{\widehat{=}}}

\begin{document}
\begin{center}
\textbf{Stochastik 1 \\ Serie 6\\}
\end{center}

\begin{flushright}
Kevin Stehn 6416016 Gruppe 3 \\
Konstantin Kobs 6414943 Gruppe 2
\end{flushright}

\section*{Aufgabe 1}
Zu zeigen ist, dass folgende Aussage gilt: $E(X+Y) = E(X) + E(Y)$\\
%Beweis dass dies stimmt:\\
%$E(X+Y) = \sum_{x\in E} \sum_{y \in E} x \cdot P(X=x) + y \cdot P(Y=y)%\\
%\\ = \sum_{x\in E} x \cdot P(Y=x) + \sum_{y \in E} y\cdot P(Y=y) 
%\\\\= E(X) + E(Y)$
Beweis:
\begin{align*}
E(X) + E(Y) &= \sum_{\omega \in \Omega} X(\omega) \cdot P({\omega}) + \sum_{\omega \in \Omega} Y(\omega) \cdot P({\omega})\\
&= \sum_{\omega \in \Omega} (X(\omega) + Y(\omega)) \cdot P({\omega})\\
&\overset{\text{Summe zweier Zgn}}{\underset{\text{ist wieder eine Zg}}{=}} E(X+Y)
\end{align*}

\section*{Aufgabe 2}
\begin{itemize}
\item[(a)]
\item[(b)]
\item[(c)]
\end{itemize}

\section*{Aufgabe 3}
Unser Modell sieht wie folgt aus: $\Omega = \{\{1,...,6\}^2\} \text{ } P =Laplaceverteilung$. Unsere Zufallsgröße $X$ bildet von $\Omega$ in die Menge $\{-3,0,2\}$ ab und beschreibt den Gewinn des Bierverkäufers. Dabei geben negative Zahlen den Verlust und positive Zahlen den Gewinn des Betrages an.
\begin{align*}
X(\omega) = \left\{
\begin{array}{l l}
-3 &\text{ für } \omega = (6,6)  \\ 
0 &\text{ für }  \omega \in \{(1,1),(2,2),(3,3),(4,4),(5,5)\}\\
2 &\text{ sonst}
\end{array}
\right.
\end{align*}
Für die Wahrscheinlichkeiten ergibt sich dann:
\begin{align*}
P(X=-3)&=\frac{1}{36}\\
P(X=0)&=\frac{5}{36}\\
P(X=2)&=\frac{30}{36}\\
\end{align*}
Der erwartete Gewinn des Verkäufers pro Spiel berechnet sich durch den Erwartungswert der Zufallsgröße $X$:
\begin{align*}
E(X) &= -3 \cdot P(X=-3) + 0 \cdot P(X=0) + 2 \cdot P(X=2)\\
&= -3 \cdot \frac{1}{36} + 0 \cdot \frac{5}{36} + 2 \cdot \frac{30}{36}\\
&= \frac{19}{12} \approx 1,58
\end{align*}
Der Verkäufer macht pro Spiel also durchschnittlich etwa $1,58$ Euro Gewinn.
\end{document}