\documentclass[10pt,a4paper]{article}
\usepackage[utf8]{inputenc}
\usepackage[german]{babel}
\usepackage{amsmath}
\usepackage{amsfonts}
\usepackage{amssymb}

\newcommand{\ent}{\mathop{\widehat{=}}}

\begin{document}
\begin{center}
\textbf{Stochastik 1 \\ Serie 1 \\}
\end{center}

\begin{flushright}
Kevin Stehn 6416016 Gruppe 3 \\
Konstantin Kobs 6414943 Gruppe 2
\end{flushright}

\section*{Aufgabe 1}
Da alle Elementarwahrscheinlichkeiten zusammen 1 sein müssen folgt also draus:\\
\[ \sum_{w=0}^\infty c * q^w = 1 \Leftrightarrow  c * \sum_{w=0}^\infty q^w = 1 \]
\[\underleftrightarrow{geom. Reihe} \text{  } \frac{c}{1-q} = 1 \]
Durch umstellen der Gleichung erhalten wir: $c = 1-q$.

\section*{Aufgabe 2}
Das Ereignis $T \ent \text{ist ein Terrorist}$ \\ 
und das Ereignis $F \ent \text{wurde festgenommen}$.
Die Wahrscheinlichkeit das ein Terrorist festgenommen wird $P(F|T) = 0,98 $,das jemand nicht festgenommen wird und kein Terrorist ist $P(F^c|T^c) = 0,99$ daraus folgt das jemand festgenommen wird der kein Terrorist ist $P(F|T^{c})= 0,01 $.\\
Die Wahrscheinlichkeit $P(T|F) \ent \text{das ein festegnommener Passagier ein Terrorist ist}$.
Wir nehmen an das: $P(T) = 0,0001\%$ und $P(T^c)= 0,9999\% $ \\
Daraus folgt:\\
$P(T|F) = \frac{P(F|T)*P(T)}{P(F|T)*P(T) + P(F|T^c) * P(T^c)} = 
\frac{0,98*0,0001}{0,98*0,0001+ 0,01*0,9999} \newline = \frac{98}{10097} \approx 9,7*10^{-3}$

\section*{Aufgabe 3}
Unser Modell ist: 
$\Omega = \{(\omega_{1},\omega_{2}) \in \{K,Z\}^2 \}$ \\
Dabei ist: $\omega_{1} \ent \text{1 .Wurf}$ und $\omega_{2} \ent \text{2. Wurf} $ und \\
$\omega_{i}=K \ent \text{Kopf} \text{ und } \omega_{i}=Z \ent \text{Zahl}$. Dadurch bekommen wir die folgenden\\ Ereignisse:\\
$A = \{(Z,K),(Z,Z)\} \text{ } P(A) = \frac{1}{2}$\\
$B = \{(K,Z),(Z,Z)\} \text{ } P(B) =\text{ } \frac{1}{2}$\\
$C = \{(K,K),(Z,Z)\} \text{ } P(C) =\text{ } \text{ } \frac{1}{2}$\\
Draus folgen die Wahrscheinlichkeiten:\\
$P(A \cap B) = \text{ }\frac{1}{4} = P(A)*P(B) \Rightarrow \text{unabhänging} \newline
P(A \cap C) = \frac{1}{4} = P(A)*P(C)  \Rightarrow \text{unabhänging} \newline
P(B \cap C) = \text{ }\frac{1}{4} = P(B)*P(C)  \Rightarrow \text{unabhänging}$\\
Für (A,B,C) müssen wir zunächst die schon oben berechneten Gleichungen überprüfen. Zusätzlich muss aber auch noch gelten: \\
$P(A \cap B \cap C) = P(A)*P(B)*P(C)$ \\
$P(A \cap B \cap C) = \frac{1}{4} \neq \frac{1}{8}  = P(A)*P(B)*P(C) \Rightarrow$  nicht unabhängig

\end{document}