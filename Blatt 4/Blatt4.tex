\documentclass[10pt,a4paper]{article}
\usepackage[utf8]{inputenc}
\usepackage[german]{babel}
\usepackage{amsmath}
\usepackage{amsfonts}
\usepackage{amssymb}

\newcommand{\ent}{\mathop{\widehat{=}}}
\newcommand{\w}{\omega}

\begin{document}
\begin{center}
\textbf{Stochastik 1 \\ Serie 4 \\}
\end{center}

\begin{flushright}
Kevin Stehn 6416016 Gruppe 3 \\
Konstantin Kobs 6414943 Gruppe 2
\end{flushright} 

\section*{Aufgabe 1}
Es wurde gezeigt, dass gilt:\\
$P(A|B) > P(A) \Rightarrow P(B|A) > P(B) \newline \Leftrightarrow 1 -P(B^c|A) > 1-P(B^c) \newline \Leftrightarrow P(B^c|A) < P(B^c)$\\
Da $P(B^c) > 0 $ gilt, gibt es immer ein $P(B^c|A)$, f\"ur das die Ungleichung erf\"ullt ist. Wir sind durch Umformung auf $P(B^c|A) < P(B^c)$ gekommen, was bedeutet, dass $B^c$ \underline{nicht} von A angezogen wird. $\square$

\section*{Aufgabe 2}
Unser Modell ist: $\Omega = \{(\w_1,\w_2) : \w_1,\w_2 \in \{1,...,6\} \}$ daf\"ur gilt\\
$\omega_1 \ent \text{1. Wurf}$ und $\w_2 \ent \text{2. Wurf}$.\\
Unser Wahrscheinlichkeitsmaß ist P = Laplacemaß.\\
Das Ereignis das die Summe beider W\"urfe 7 ist:\\
$A = \{\w \in \Omega : \w_1 + \w_2 = 7 \}$.\\
Und das Ereignis das der 1. Wurfe eine 6 ist:\\
$B = \{\w \in \Omega : \w_1 = 6$.
Die Wahrscheinlichkeiten der Ereignisse sind:\\
$P(A) = \frac{1}{6} \text{ und } P(B) = \frac{1}{6}$ \\
F\"ur die stochastische Unabh\"angigkeit folgt daraus:\\
$P(A \cap B) = \frac{1}{36} = P(A) \cdot P(B) \Rightarrow \text{unabh\"agig}$

\section*{Aufgabe 3}
\begin{itemize}
\item[a)]Da die Summe der Verteilung 1 ergeben muss lässt sich q wie folgt berechen.\\
$1 = \frac{1}{4}+\frac{1}{6}+\frac{1}{4}+q$. Durch Umstellen der Gleichung erhalten wir:\\\\
$q = 1-\frac{1}{4}-\frac{1}{6}-\frac{1}{4} \newline
q = \frac{1}{3}$
\item[b)]Damit X und Y unabhängig sind, muss gelten:\\
\begin{equation}
  \left.
  \begin{aligned}
     \frac{1}{4} = P(X=1,Y=1) = P(X=1) \cdot P(Y=1)  \\
     \frac{1}{4} = P(X=1, Y=2) = P(X=1) \cdot P(Y=1)  
   \end{aligned}
   \right\}
  	\text{\scriptsize Damit dies gilt, muss P(Y=1)=P(Y=2) sein}
\end{equation}\\
\begin{equation}
  \left.
  \begin{aligned}
     \frac{1}{6} = P(X=2,Y=1) = P(X=2) \cdot P(Y=1)  \\
     \frac{2}{6} = P(X=2, Y=2) = P(X=2) \cdot P(Y=1)  
   \end{aligned}
   \right\}
   \text{\scriptsize Wiederspruch hierzu, da zwei verschiedene Wk's herauskommen}
\end{equation}\\
Egal, wie das Wahrscheinlichkeitsmaß für X und Y aussieht, kann das Gleichungssystem nicht stimmen. Deshalb sind X und Y abhängig voneinander. 
\end{itemize}
\end{document}