\documentclass[10pt,a4paper]{article}
\usepackage[utf8]{inputenc}
\usepackage[german]{babel}
\usepackage{amsmath}
\usepackage{amsfonts}
\usepackage{amssymb}

\newcommand{\ent}{\mathop{\widehat{=}}}
\newcommand{\w}{\omega}

\begin{document}
\begin{center}
\textbf{Stochastik 1 \\ Serie 4 \\}
\end{center}

\begin{flushright}
Kevin Stehn 6416016 Gruppe 3 \\
Konstantin Kobs 6414943 Gruppe 2
\end{flushright}

\section*{Aufgabe 1}

\section*{Aufgabe 2}
Unser Modell ist: $\Omega = \{(\w_1,\w_2) : \w_1,\w_2 \in \{1,...,6\} \}$ daf\"ur gilt\\
$\omega_1 \ent \text{1. Wurf}$ und $\w_2 \ent \text{2. Wurf}$.
Das Ereignis das die Summe beider W\"urfe 7 ist:\\
$A = \{\w \in \Omega : \w_1 + \w_2 = 7 \}$.\\
Und das Ereignis das der 1. Wurfe eine 6 ist:\\
$B = \{\w \in \Omega : \w_1 = 6$.
Die Wahrscheinlichkeiten der Ereignisse sind:\\
$P(A) = \frac{1}{6} \text{ und } P(B) = \frac{1}{6}$ \\
F\"ur die stochastische Unabh\"angigkeit folgt daraus:\\
$P(A \cap B) = \frac{1}{36} = P(A) \cdot P(B) \Rightarrow \text{unabh\"agig}$

\section*{Aufgabe 3}
\end{document}