\documentclass[10pt,a4paper]{article}
\usepackage[utf8]{inputenc}
\usepackage[german]{babel}
\usepackage{amsmath}
\usepackage{amsfonts}
\usepackage{amssymb}

\begin{document}
\begin{center}
\textbf{Stochastik 1 \\ Serie 1 \\}
\end{center}

\begin{flushright}
Kevin Stehn 6416016 Gruppe 3 \\
Konstantin Kobs
\end{flushright}

\section*{Aufgabe 1}
\begin{itemize}
\item[(a)] $(A \cup B \cup C)^{c}$
% Es ist das Komplement der Gesamtmenge und nicht die Gesamtmenge der Komplemente. ;)
\item[(b)] 
$(A \cap B \cap C^{c}) \cup (A \cap B^{c} \cap C) \cup (A^{c} \cap B \cap C)$
% Die Mengen in den Klammern müssen geschnitten werden
\end{itemize}

\section*{Aufgabe 2}
\begin{itemize}
\item[(a)]
Zu zeigen: $\frac{1}{12} \leq P(A \cap B) \leq \frac{1}{3}$

Die beiden Teilmengen A und B können nicht disjunkt sein, denn sonst wäre $P(A) \cap P(B) \leq 1$, was aber nicht der Fall ist. Sie müssen sich also in mindestens einem Ergebnis überschneiden. Nun kann man zwei Extrema betrachten:

1. Die beiden Mengen haben so viele Elemente wie möglich gemeinsam.\\
2. Die beiden Mengen haben so wenig Elemente wie möglich gemeinsam.

Zu 1: Haben A und B so viele Elemente wie möglich gemeinsam, so ist B eine Teilmenge von A, weil B eine geringere Wahrscheinlichkeit hat als A. Dadurch kann A nicht in B liegen. Wenn alle Elemente von B auch in A liegen, dann gilt für die Wahrscheinlichkeit $P(A \cap B) \leq P(B) = \frac{1}{3}$. Dies ist also die obere Grenze der Wahrscheinlichkeit. $\square$

Zu 2: Wenn so wenig Elemente wie möglich in A und B gleichzeitig liegen, dann können die Wahrscheinlichkeiten nur folgendermaßen verteilt sein:
A nimmt $\frac{3}{4}$ der Gesamtwahrscheinlichkeit ein. Also kann B nur noch $\frac{1}{4}$ ``auffüllen". Somit bleibt eine Überschneidung von mindestens $P(A \cap B) \geq P(B) - P(A^c) = \frac{1}{3} - \frac{1}{4} = \frac{1}{12}$. $\square$

\item[(b)]
% WAS MEINEN DIE MIT BEISPIELEN!?!?!?!
\end{itemize}

\section*{Aufgabe 3}
\begin{itemize}
\item[(a)] Wenn wir davon ausgehen können, dass das Losverfahren zur Platzbestimmung gleichverteilt ist, ist es sinnvoll, vom Laplacemaß auszugehen.\\
Als Modell: $ \Omega = \{ (\omega_{1},\omega_{2}, \omega_{3}, ..., \omega_{12}) | \omega_{i} \in\{1,...,12\} \wedge \omega_{i} \neq \omega_{j} \text{ für alle } i \neq j \} $, wobei 1 = Herr Meyer ist und 2 = Frau Müller.

\item[(b)]
Da der Tisch rund ist, müssen wir die Personen an den Enden des Tupels ebenfalls als nebeneinandersitzend ansehen.\\
$A = \{(\omega_1, \omega_2, ..., \omega_{12}) \in \Omega | \exists i \in \{1, ...,11\} : (\omega_i = 1 \wedge \omega_{i+1} = 2) \vee (\omega_i = 2 \wedge \omega_{i+1} = 1) \vee (\omega_1 = 1 \wedge \omega_{12} = 2) \vee (\omega_1 = 2 \wedge \omega_{12} = 1) \}$

\item[(c)]
Die Mächtigkeit von $\Omega$ ist $12!$, da wir jeder Person einen Platz zulosen und nicht eine Person an mehreren Plätzen sitzen haben können.\\
Die Mächtigkeit von $A$ ist folgendermaßen berechenbar: Wir setzen Herrn Meyer und Frau Müller nebeneinander und die restlichen Personen losen wir wieder zu. Das sind also $10!$. Da wir Meyer und Müller aber auch tauschen können, verdoppelt sich die Anzahl der Möglichkeiten. Nun können wir die beiden 12 mal um einen Platz nach links oder rechts versetzen und das gleiche Spiel nach jedem Platzwechsel erneut machen. Damit ergibt sich als Mächtigkeit von A $2 \cdot 12 \cdot 10!$.

Insgesamt gilt also:\\
$P(A) = \frac{|A|}{|\Omega|} = \frac{2 \cdot 12 \cdot 10!}{12!} = \frac{2}{11} = 0,\overline{18}$

\end{itemize}
\end{document}