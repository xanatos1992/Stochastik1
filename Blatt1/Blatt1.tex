\documentclass[10pt,a4paper]{article}
\usepackage[utf8]{inputenc}
\usepackage[german]{babel}
\usepackage{amsmath}
\usepackage{amsfonts}
\usepackage{amssymb}

\begin{document}
\begin{center}
\textbf{Stochastik 1 \\ Serie 1 \\}
\end{center}

\begin{flushright}
Kevin Stehn 6416016 Gruppe 3 \\
Konstantin Kobs
\end{flushright}

\section*{Aufgabe 1}
\begin{itemize}
\item[(a)] $(A \cup B \cup C)^{c}$
% Es ist das Komplement der Gesamtmenge und nicht die Gesamtmenge der Komplemente. ;)
\item[(b)] 
$(A \cap B \cap C^{c}) \cup (A \cap B^{c} \cap C) \cup (A^{c} \cap B \cap C)$
% Die Mengen in den Klammern müssen geschnitten werden
\end{itemize}

\section*{Aufgabe 2}
\begin{itemize}
\item[(a)]
Zu zeigen: $\frac{1}{12} \leq P(A \cap B) \leq \frac{1}{3}$

\item[(b)]
\end{itemize}

\section*{Aufgabe 3}
\begin{itemize}
\item[(a)] Wenn wir davon ausgehen können das jede Person die gleiche Wahrscheinlichkeit hat einen Platz zu bekommen, also eine Gleichverteilung vorliegt ist es sinnvoll Laplacemaß zu nehmen.\\
Als Modell: $ \Omega = \{(\omega_{1},\omega{2}) | \omega_{1},\omega{2} \in\{1,...,12\} \} $, wobei 1 = Herr Meyer ist und 2 = Frau Müller die Restlichen $\geq$ 3 entsprechen die anderen Personen. 

\item[(b)]
A = \{(1,2)\}

\item[(c)]
$P(A) = \frac{\mid A \mid}{\mid \Omega \mid} = \frac{1}{78}$ %ist das richtig? 
\end{itemize}
\end{document}