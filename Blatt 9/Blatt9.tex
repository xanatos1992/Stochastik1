\documentclass[10pt,a4paper]{article}
\usepackage[utf8]{inputenc}
\usepackage[german]{babel}
\usepackage{amsmath}
\usepackage{amsfonts}
\usepackage{amssymb}

%Zur Code Darstellung
\usepackage{listings}
\usepackage{color}
\definecolor{dkgreen}{rgb}{0,0.6,0}
\definecolor{gray}{rgb}{0.5,0.5,0.5}
\definecolor{mauve}{rgb}{0.58,0,0.82}

\lstset{frame=tb,
  language=Java,
  aboveskip=3mm,
  belowskip=3mm,
  showstringspaces=false,
  columns=flexible,
  basicstyle={\small\ttfamily},
  numbers=none,
  numberstyle=\tiny\color{gray},
  keywordstyle=\color{blue},
  commentstyle=\color{dkgreen},
  stringstyle=\color{mauve},
  breaklines=true,
  breakatwhitespace=true,
  tabsize=3,
  literate=	{ö}{{\"o}}1
           		{ä}{{\"a}}1
           		{ü}{{\"u}}1
}

\newcommand{\ent}{\mathop{\widehat{=}}}

\begin{document}
\begin{center}
\textbf{Stochastik 1 \\ Serie 9 \\}
\end{center}

\begin{flushright}
Kevin Stehn 6416016 Gruppe 3 \\
Konstantin Kobs 6414943 Gruppe 2
\end{flushright}

\section*{Aufgabe 1}
\begin{lstlisting}
// Die Anzahl der Durchführungen des Versuches
var durchfuehrungen = 10000;

/**********************************************
	Generiert eine Zahl zwischen 1 und 5,
	"zieht" also die Kugel mit Zurücklegen
***********************************************/
function zieheKugel(){
	return Math.floor(Math.random()*5 + 1);
}

/**********************************************
	"Zieht" drei Kugeln und gibt die kleinste der
	gezogenen Zahlen zurück.
***********************************************/
function dreiKugelnMinimum(){
	return Math.min(zieheKugel(), zieheKugel(), zieheKugel());
}

var summe = 0;
// Den Versuch immer wieder durchführen
for(var i = 0; i < durchfuehrungen; ++i){
	// Summe aller Einzelergebnisse berechnen
	summe += dreiKugelnMinimum();
}
// Erwartungswert berechnen
var erwartungswert = summe/durchfuehrungen;

console.log("Bei " + durchfuehrungen + " Durchführungen des Versuches ergab sich für den Erwartungswert " + erwartungswert);
// Gibt zum Beispiel "Bei 10000 Durchführungen des Versuches ergab sich für den Erwartungswert 1.8078" aus.
\end{lstlisting}

\section*{Aufgabe 2 und 3}
\begin{itemize}
\item[(a)] Das Modell sind Zufallsgrößen von 1 bis 10.000 $X_1,...,X_{10000}$. Diese sind stoch. unabhängig und identisch verteilt mit $P(X_i = 0) = \frac{1}{10} \text{ und }\\
P(X_i=1) = \frac{9}{10}$. Dabei gilt:\\
$\{X_i = 0\} \ent \text{ Passagier i erscheint nicht}\\
\{X_i=1\} \ent \text{ Passagier i erscheint}$
\item[(b)] Das Ereignis, dass mindestens ein Passagier nicht mit kann ist $\{ \sum\nolimits_{i=1}^{10000} X_i > 9060\}$.
\item[(c)] Um den zentralen Grenzwertsatz verwenden zu können, benötigen wir $\mu = E(X_i)$ und $\sigma^2 = Var(X_i)$.\\
\begin{align*}
E(X_i) &= 0*\frac{1}{10} + 1* \frac{9}{10} = \frac{9}{10}\\
&= 0,9 \\
Var(X_i) &= E(X_i^2)-(E(X_i))^2\\
&= (0^2*\frac{1}{10}+1^2*\frac{1}{10})-\frac{9}{10}^2\\
&= \frac{9}{100} = 0,09
\end{align*}
Damit ist $\mu = 0,9$ und $\sigma^2 = 0,09$. Die Wahrscheinlichkeit lässt sich wie folgt berechnen:\\
\begin{align*}
P(\sum\nolimits_{i=1}^{10000} X_i > 9060) &= 1 - P(\sum\nolimits_{i=1}^{10000} X_i \leq 9060)\\
&= 1-P(\sum\nolimits_{i=1}^{10000} X_i-9000 \leq 60)\\
&= 1-P(\frac{\sum\nolimits_{i=1}^{10000} X_i-9000}{30} \leq 2)\\
&\approx 1-\Phi(2) \approx \frac{57}{2500}\\
&= 0,0228
\end{align*}
Das heißt, zu einer Wahrscheinlichkeit von 2,28\% muss mindestens ein Passagier an Land bleiben.
\end{itemize}

\end{document}