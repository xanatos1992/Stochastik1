\documentclass[10pt,a4paper]{article}
\usepackage[utf8]{inputenc}
\usepackage[german]{babel}
\usepackage{amsmath}
\usepackage{amsfonts}
\usepackage{amssymb}

%Zur Code Darstellung
\usepackage{listings}
\usepackage{color}
\definecolor{dkgreen}{rgb}{0,0.6,0}
\definecolor{gray}{rgb}{0.5,0.5,0.5}
\definecolor{mauve}{rgb}{0.58,0,0.82}

\lstset{frame=tb,
  language=Java,
  aboveskip=3mm,
  belowskip=3mm,
  showstringspaces=false,
  columns=flexible,
  basicstyle={\small\ttfamily},
  numbers=none,
  numberstyle=\tiny\color{gray},
  keywordstyle=\color{blue},
  commentstyle=\color{dkgreen},
  stringstyle=\color{mauve},
  breaklines=true,
  breakatwhitespace=true
  tabsize=3
}

\newcommand{\ent}{\mathop{\widehat{=}}}

\begin{document}
\begin{center}
\textbf{Stochastik 1 \\ Serie 9 \\}
\end{center}

\begin{flushright}
Kevin Stehn 6416016 Gruppe 3 \\
Konstantin Kobs 6414943 Gruppe 2
\end{flushright}

\section*{Aufgabe 1}
\begin{lstlisting}
import java.util.ArrayList;
import java.util.Random;

public class MonteCarloVerfahren
{
	/**
	 * Ziehen von drei Kugeln mit nummern von 1-5 davon den geringsten wert und
	 * erwartungswert berechnen
	 * 
	 * @param x
	 *            wie haeufig ziehen
	 * @return
	 */
	public static double monteCarlo(int x)
	{
		double erg = 0;
		ArrayList<Integer> minWerte = new ArrayList<Integer>();
		Random random = new Random();
		
		int[] kugeln = new int[3];
		
		//X mal Drei Kugeln Ziehen und Min Wert Herrausfinden
		for (int d = 1; d < x; d++)
		{
			// Drei mal Ziehen
			for (int i = 0; i < 3; i++)
			{
				kugeln[i] = random.nextInt(6 - 1) + 1;
			}

			int minWert = Integer.MAX_VALUE;
			// Kleinsten Wert finden
			for (int i = 0; i < 3; i++)
			{
				if (kugeln[i] < minWert)
				{
					minWert = kugeln[i];
				}
			}
			
			minWerte.add(minWert);
		}
		
		double sum = summeRechnen(minWerte);
		
		erg = sum / x;
		return erg;
	}
	
	public static int summeRechnen(ArrayList<Integer> menge)
	{
		int erg = 0;
		for(Integer i : menge)
		{
			erg = erg+i;
		}
		return erg;
	}

	public static void main(String[] args)
	{
		//Ausfuehrungen
		int x = 10000;
		System.out.println("Monte Carlo Verfahren mit " + x + " Ausfuehrungen:");
		double erwartungswert = monteCarlo(x);
		System.out.println("Erwartungs Wert: " + erwartungswert);
	}
}

\end{lstlisting}
\section*{Aufgabe 2 und 3}
\begin{itemize}
\item[(a)] Das Model sind Zufallsvariablen von 1 bis 10.000 $X_1,...,X_10000$. Diese sind stoch. unabhängig und identisch verteilt mit $P(X_i = 0) = \frac{1}{10} \text{ und }\\
P(X_i=1) = \frac{9}{10}$. Dabei gilt:\\
$\{X_i = 0\} \ent \text{ Passagier i erscheint nicht}\\
\{X_i=1\} \ent \text{ Passagier i erscheint}$
\item[(b)] Das Ereignis das min. ein Passagier nicht mit kann ist $\{ \sum\nolimits_{i=1}^{10000} X_i > 9060\}$.
\item[(c)] Um den zentralen Grenzwertsatz Verwenden zu können benötigen wir\\
 $\mu = E(X_i)$ und $\sigma^2 = Var(X_i)$.\\
 $E(X_i) = 0*\frac{1}{10} + 1* \frac{9}{10} = \frac{9}{10} = 0,9 \\
 Var(X_i) = E(X_i^2)-(E(X_i))^2 = (0^2*\frac{1}{10}+1^2*\frac{1}{10})-\frac{9}{10}^2 = \frac{9}{100} = 0,09$. \\
 Damit ist $\mu = 0,9$ und $\sigma^2 = 0,09$. Die Wahrscheinlichkeit lässt sich wie folgt berechnen:\\
$P(\sum\nolimits_{i=1}^{10000} X_i > 9060) = 1 - P(\sum\nolimits_{i=1}^{10000} X_i \leq 9060)\\\\
= 1-P(\sum\nolimits_{i=1}^{10000} X_i-9000 \leq 60\\\\
= 1-P(\frac{\sum\nolimits_{i=1}^{10000} X_i-9000}{30} \leq 2) \\\\
\approx 1-\Phi(2) \approx \frac{57}{2500} = 0,0228$.\\
Das heißt zu einer Wahrscheinlichkeit von 2,28% muss min. ein Passagier an Land bleiben.
\end{itemize}

\end{document}